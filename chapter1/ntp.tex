\subsection{The NTP algorithm}
\label{sec:ntp}

The NTP protocol\cite{mills} was originally devlopeded in 1985 by David L.Mills. It is used to synchronize clocks over a network by calculating the master offset using the round trip delay. NTP is still in use today, the latest RTC is from June 2010\footnote{\url{https://tools.ietf.org/html/rfc5905}}.

The NTP algorithm uses four different timestamps to calculate the round trip delay. 

\begin{itemize}
  \item[] \textbf{t0} is the time of the request packet transmission
  \item[] \textbf{t1} is the time of the request packet reception
  \item[] \textbf{t2} is the time of the response packet transmission
  \item[] \textbf{t3} is the time of the response packet reception.
\end{itemize}
\label{fig:ntpvars}

The timestamps of t0 and t3 is set by the sender and t1 and t2 by the reciever. The sender then calculates the round trip delay, $\delta$. 

\begin{displaymath}
	\delta = (\text{t3} - \text{t0}) - (\text{t2} - \text{t1})
\end{displaymath}
\label{fig:ntpdelta}

NTP assumes that the delay of the sender and the reciever is equal, and thus calculates the master offset, $\theta$,  as below. 

\begin{displaymath}
	\theta = \frac{(\text{t3} - \text{t0}) - (\text{t2} - \text{t1})}{2}
\end{displaymath}
\label{fig:ntpmo}

