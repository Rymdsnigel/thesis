\subsection{The NTP protocol algorithm}
\label{sec:ntp}

The NTP protocol was originally devlopeded in 1985 by David L.Mills %SOURCE
It is used to synchronize clocks over an network by calculating the master offset using the round trip delay.

\begin{figure}[h!]
	\begin{displaymath}
		\delta = (\text{t3} - \text{t0}) - (\text{t2} - \text{t1})
	\end{displaymath}
	\caption{Calculating the round trip delay}
	\label{fig:ntpdelta}
\end{figure}


\begin{figure}[h!]
\begin{itemize}
  \item[] \textbf{t0} is the time of the request packet transmission
  \item[] \textbf{t1} is the time of the request packet reception
  \item[] \textbf{t2} is the time of the response packet transmission
  \item[] \textbf{t3} is the time of the response packet reception.
\end{itemize}
\caption{Variables for calculating round-trip delay time}
\label{fig:ntpvars}
\end{figure}

\begin{figure}[h!]
	\begin{displaymath}
		\delta = \frac{(\text{t3} - \text{t0}) - (\text{t2} - \text{t1})}{2}
	\end{displaymath}
	\caption{Calculating the master offset}
	\label{fig:ntpmo}
\end{figure}



NTP is one of the oldest protocols still in use, thus the algorithm take into account the processing between reception and transmisson of a package. This difference would be significant in computers in the 80:s and 90:s but on the computer tested on %SOURCE my computer specs 
this will be sometime around 1 millisecond. 
