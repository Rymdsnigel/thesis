\chapter{About synchronization}

\begin{quotation}
\bf You may delay, but time will not.
\rm \center \em - Benjamin Franklin
\end{quotation}

\section{Background}

Because projectors and screens have a limited resolution, achieving a specific higher resolution requires combining several screens or projectors. This leads to a need to synchronize the graphics-rendering of an arbitrary number of computers, connected to these screens or projectors. This thesis will explore the possibility of synchronizing rendering.

The work of this thesis will ultimately be used in a system to synchronize the rendering in a platform for graphics visualisation written in C++, OpenGL3 and GLSL. According to the specification, specific events also needs to be able to be sent to the clients, and the clients should perform specific animations on recieving these events.  

\subsection{When is the rendering synchronized?}

Research has been made into the maximum amount of delay acceptable for the human eye between Audio and Video\footnote{\url{http://tech.ebu.ch/docs/r/r037.pdf}, EBU}, as this is important in the broadcasting of television. No research into what could be ''acceptable'' asychronization for the human eye in video to video synchronization will be made in this thesis, it will be assumed that it is a soft Real Time system and that the delay between the video should be as short as possible, an best effort system. For the sake of clarity we discuss delays in milliseconds.

\subsection{What needs to be synchronized?}

Since the clients rendering the animations will be distributed on different computers it cannot be assumed that the clocks on these computers are in sync with each other, this needs to be solved, the clients needs to have the same perception of time. When recieving a message from the server, all clients needs to synchronize their animations with each other, thus the clients needs to have some kind of concept of their, or the other clients latencies. 

\subsection{What is specific for synchronizing rendering?}
There are a few tools available when working with rendering, there is first of all the possibility to change the speed and start time of animations but also the possibility to frameskip. There is also the possibility to framestep. 

\section{Synchronizing clocks over an network}
The problem of synchronizing computer clocks over an network has been investigated since the early days of networks. There are a few well known algoritms, these are the ones looked into in the making of this thesis. 
\subsection{The NTP algorithm}
\label{sec:ntp}

The NTP protocol\cite{mills} was originally devlopeded in 1985 by David L.Mills. It is used to synchronize clocks over a network by calculating the master offset using the round trip delay. NTP is still in use today, the latest RTC is from June 2010\footnote{\url{https://tools.ietf.org/html/rfc5905}}.

The NTP algorithm uses four different timestamps to calculate the round trip delay. 

\begin{itemize}
  \item[] \textbf{t0} is the time of the request packet transmission
  \item[] \textbf{t1} is the time of the request packet reception
  \item[] \textbf{t2} is the time of the response packet transmission
  \item[] \textbf{t3} is the time of the response packet reception.
\end{itemize}
\label{fig:ntpvars}

The timestamps of t0 and t3 is set by the sender and t1 and t2 by the reciever. The sender then calculates the round trip delay, $\delta$. 

\begin{displaymath}
	\delta = (\text{t3} - \text{t0}) - (\text{t2} - \text{t1})
\end{displaymath}
\label{fig:ntpdelta}

NTP assumes that the delay of the sender and the reciever is equal, and thus calculates the master offset, $\theta$,  as below. 

\begin{displaymath}
	\theta = \frac{(\text{t3} - \text{t0}) - (\text{t2} - \text{t1})}{2}
\end{displaymath}
\label{fig:ntpmo}


\subsection{The Berkley algorithm}
\label{sec:berkley}
The Berkley algorithm was written by Gusella and Zatti in 1989. A simplification of the steps of the algorithm is shown below. 

\begin{enumerate}
\item A master polls slaves, the slaves reply with their time.
\item The master uses the round-trip time of the messages to estimate the time of each slave and the master's own time.
\item The master calculates an average of the clock times, ignoring extreme values.
\item The master sends the slaves their delta, which can be positive or negative. 
\end{enumerate}
\label{fig:mifare-auth}

The delta value is used by each slave to adapt their time to the choosen master time. 

%needs references

\subsection{Cristians algorithm}
Cristians algorithm\cite{cristian}, developed in 1989 by Flaviu Cristian, is used for synchronizing clocks by calculating the round trip time. It is a probabilistic algorithm\cite{cristian}, meaning that it it delivers a better accuracy the shorter the round trip delay time is. 

The algorithm works as follows. The server sends a message to the client, and the client replies. When the server recieves the reply from the client it calculates the round trip delay time, that is, the time passed between sending the request from the server and recieving the reply from the client divided by 2. This assumes that the latency of the server and the client is equal.

The server then sends the sum of its own time added with the round trip delay time to the client, and the client sets this as its own time. 

\begin{figure}[h!]
	\begin{displaymath}
		\text{Client time} = \text{Server time} + \text{Round trip delay time}
	\end{displaymath}
	\caption{Setting the time on the Client}
	\label{fig:crist}
\end{figure}


\subsection{PTP and GPS}

The Precision Time Protocol, abbriviates PTP, is a protocol used for high accuracy synchronization in time critical systems. The first RTC is from 2002. Although PTP achieves very high synchronization it was believed in the making of this thesis that synchronizing rendering would not require that kind of precision. 

One other solution for synchronizing the time of distributed applications is to use GPS. This has not been investigated further in this thesis since this would require that all clients had a GPS-sender/reciever, something that seemed improbable. 


\section{Problem definition}
\label{sec:problem_definition}

\begin{itemize}
  \item Can we synchronize the rendering on an arbitrary number of computers?
  \begin{itemize}
    \item How can we solve the messaging of synchronization-events?
  	\item When and how often do we need to synchronize?
    \item What techniques are available?
    \item Which way gives the tightest synchronization?
    \item Which way is the most efficient?
  	\item How can this be optimized?
  \end{itemize}
\end{itemize}
