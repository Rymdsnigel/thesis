\section{Client latency}

\subsection{Setting latency of client}
The latency of each client is calculated on the server and sent to the client in an latency\_update\_event, along with the maximum latecy. The maximum latency is the latency of the client with the greatest latency. The client then calculates the latency it should apply by subtracting its own latency from the maximum latency.

\begin{figure}[h!]
	\begin{displaymath}
		\text{applied\_latency} = \text{maximum\_latency} - \text{latency}
	\end{displaymath}
	\caption{Calculating the applied latency}
	\label{fig:applatency}
\end{figure} 

\subsection{Delaying animation start based on latency}
Before placing a new render event on the queue the client sleeps, using a greenlet.sleep, for the number of milliseconds specified by its applied latency.

\subsection{Skipping frames if delay is to long}
If the client latency is to long we should skip ahead instead.
