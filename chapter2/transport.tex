\section{Communication}

\subsection{Protocols}
The choice of TCP-sockets as opposed to UDP-sockets was made at an early stage of development. Although UDP is typically the ideal choice for time critical applications, it was belived that customizing the needed controlmechanisms would be outside the scope of this thesis. 

The choice of TCP presented itself as an issue when it was discovered that TCP:s message buffering, using Nagels algoritm, generated a general delay in messaging, a delay of 20 ms both from the server as from the clients. This was discovered by measuring the delays of the applications over localhost, where networkdelays should be close to 0. It also ment that the more than one json object could be put on the queue of recieved events, which lead to a json-parsing error when trying to load the objects. These error and the buffer delays where removed by disabling Nagels algoritm\footnote{\url{http://stackoverflow.com/questions/8617809/unstable-tcp-receive-times}}.

%source nagels algoritm

\begin{figure}[h!]
\centering
\texttt{self.s.setsockopt(socket.IPPROTO\_TCP, socket.TCP\_NODELAY, 1)}
\caption{Disabling Nagels algorithm on a socket}
\end{figure}

\subsection{Sockets}
The demo server communicates with the clients via gevent sockets since the sockets needed to be threaded to not block the other processes. 

% ref to section about greenlets.
% explaining gevent sockets as opposed to regular sockets. 

\subsection{Replacing the networklayer}
Functional cohesion has been strived for so that the transport-part of the code could easily be replaced by for example an implementation using UDP or implementation of a ready solution such as redis. Though this has not been entirely accomplished %in which ways is it not.




